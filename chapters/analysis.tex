\mychapter{3}{Input variable analysis}
\label{sec:unchapitre}

A large set of variables is available from CMS data, they describe various aspect of photons and will be used to
distinguish between prompt and fragmentation photon.\\
To perform classification a multivariate analysis will be implemented, but MVA training can be time consuming and the "curse of
dimensionality" \footnote{Curse of dimensionality refers to problems that commonly arise when analyzing high-dimensionality data.
Increasing dimensionality lead to an increase of volume and so tends to scatter data points.} forces us to select the shortest possible input set.\\

Variables with most differences of shape for background and signal will be the most relevant for the MVA classification.
%\begin{description}
%    \item [Background vs Signal discrimination :] Variables with most differences of shape for background and signal will be picked.
%    \item [Low correlation between variables :] Needed in order to reduce redundancy of input data and thus will permit
%    to reduce MVA complexity (for example number of hidden neurons in ANN).
%\end{description}



The MVA will be trained with MC simulation for the signal sample and with the real data for the background sample.
Indeed we trust MC simulation for the signal sample (\textgamma+jet events)
but on the contrary MC backgound (multi-jet) may not be accurate (by not considering all the contributing processes)
and would give us low statistics leading to fluctuations.
For this reason, a control region enriched in background multijet events (called sidebands in the following) has to be
defined.
Events in the sidebands will be used to extract the expected distribution of discriminant variables for background, and use them in the training of the multivariate analysis. The results will be then used in a signal region, in order to extract the signal directly from data.

\section{Background vs Signal discrimination}

The choice of discriminating variables is done by looking at their shape for background and signal, both taken from MC simulation.\\
Fig. (\ref{NHiso_photon_dataVsMCbg}) shows an example of MC background simulation and sidebands data comparison for \emph{neutral hadron isolation} variable. \\

\begin{figure}[h!]
  \centering
  \includegraphics[width=0.7\textwidth]{NHiso_photon}\\[1cm]
  \caption{Top plot : Neutral hadron isolation variable for background (blue histogram) and signal MC (red histogram) and real data superimposed (empty circles). Normalized to integrated luminosity of $~36fb^{-1}$\\Bottom plot : Ratio of total expected events from MC (background+signal) over real data (blue cross) fitted by a constant (red line).}
  \label{NHiso_photon_dataVsMCbg}
\end{figure}

Since the background is extracted from a data control region for the final analysis, a cross-check of the variables
shape has to be done between data (in the sidebands) and MC to validate this control region.\\

\section{Variable correlations}

It is interesting to look at the correlations between the variables considered in the multivariate analysis, which can also be a source of discrimination.
Fig. (3.4) shows an example of the correlation values for the background MC.

\begin{figure}[ht!]
  \centering
  \includegraphics[width=0.7\textwidth]{corrMatrix_bgMC}\\[1cm]
  \caption{Correlation matrix for background MC, each line or column represent a variable.}
  \label{corrMatrix_bgMC}
\end{figure}

\section{Background-enriched control region definition}

Because we use the distribution of the data in the control region as a proxy for their distribution in the signal region, we need to make sure that the variable for the sideband definition has low correlations with the ones used in the multivariate analysis.
By looking at the correlation matrix fig. (\ref{corrMatrix_bgMC}) we can see that \emph{charged hadron isolation} (:= $I_{CH}$) is one good candidate and so will be used next for the sideband definition.\\
The sideband defined in order to find a good compromise between background purity and number of events on data.

%An MVA will be performed with real data for the background, thereby a sideband (background enriched region in the data
%sample) has to be defined on a low-correlated variable.\emph{Charged hadron isolation} fig. (\ref{sideband}) has been
%chosen and the sideband defined in order to find a good compromise between background purity and number of events on data.
\begin{description}
	\item [Sideband definition]
	\begin{description}
    	\item 2.3 < $I_{CH}$ < 15.
    	\item Background purity = 95.00 \%
		\item Number of events = $7.59*10^5$
	\end{description}
\end{description}

%Then a signal enriched region has been defined on the same variable, this cut will be applied on the signal data sample.
%\begin{description}
%	\item [Signal region definition]
%	\begin{description}
%    	\item \emph{Charge hadron isolation} < 2.
%    	\item Signal purity = ?? \%
%		\item Number of events = ??
%	\end{description}
%\end{description}

Fig. (\ref{sideband}) sketches the definition of the sidebands and the signal region.

\begin{figure}[ht!]
  \centering
  \includegraphics[width=0.7\textwidth]{sideband}\\[1cm]
  \caption{Charged hadron isolation for background MC (blue histogram), signal MC (red histogram) and real data
  superimposed (empty circles). Normalized to integrated luminosity of $~36fb^{-1}$\\On top of this is the sideband
  definition (red shaded area) and the signal region definition (blue shaded area)\\Bottom plot : Ratio of real data over total expected events from MC (background+signal) (blue cross), fitted by a constant (red line).}
  \label{sideband}
\end{figure}

For cross-check, we compare the variables shape for background MC in the signal region and DATA in the sideband region.
Fig. (\ref{MCbg_all_NHiso_photon}) shows an example of a comparison between \emph{neutral hadron isolation} for data in
the sideband region and background Monte-Carlo.\\
We can see a good agreement for MC and real data, except for a small trend in the high energy range probably due to the low statistics.
It can also be noticed that using the sidebands sensibly increases the available background statistics, with respect to
the simulation.
\begin{figure}[h!]
  \centering
  \includegraphics[width=0.7\textwidth]{MCbg_all_NHiso_photon}\\[1cm]
  \caption{Neutral hadron isolation for background MC (blue histogram) and real data in the sideband superimposed (empty circles). The integral of both distribution are normalized to unity.\\Bottom plot : Ratio of real data over background MC (blue cross) fitted by a constant (red line).}
  \label{MCbg_all_NHiso_photon}
\end{figure}

%%% Local Variables: 
%%% mode: latex
%%% TeX-master: "isae-report-template"
%%% End: 
