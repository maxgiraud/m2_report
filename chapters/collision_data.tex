\mychapter{2}{Collision data}
\label{sec:unchapitre}

In this chapter will be described the data and simulations used for this study. Section 2.3 describes the input
variables that have used for photon identification purpose.

\section{Monte-Carlo simulation}

?

\section{CMS data}

Run 2 at $\sqrt{s} = 13 TeV$ for an integrated luminosity of $~ 36 fb^{-1}$
number of events ? slide de hugues ?

\section{MVA variables}

We are interested in prompt photons comming from the hard scattering, for identifying thes photons we need to perform a
multivariate analysis.
For this work we use multiple variables representing various aspects of reconstructed photons :

\begin{description}
    \item [Isolation variables] represent additional objects (photons,charged hadron and neutral hadron) reconstructed
    in a $\Delta R$ radius cone around the processed photon. These variables permit to discriminate between isolated
    prompt photons and neutral pions within a jet.
    \begin{description}
	    \item [Charged Hadron isolation (CHiso) : ] $I_{cha} = \sum_{cha_i}^{\Delta R}{p_{T,cha_i}}$ \\
            $cha_i$ corresponds to reconstructed charged hadron.
	    \item [Neutral Hadron isolation (NHiso) : ] $I_{neu} = \sum_{neu_i}^{\Delta R}{p_{T,neu_i}}$ \\
            $neu_i$ corresponds to reconstructed neutral hadron.
        \item [Photon isolation (Photoniso) : ] $I_\gamma = \sum_{\gamma_i}^{\Delta R}{p_{T,\gamma_i}}$ \\
            $\gamma_i$ corresponds to reconstructed photons, the sum doesn't account for the $p_T$ of the processed
            photon. (parler du pile-up avec $\rho$ ?)
	\\
    \end{description}
    \item [Shape variables] represent deposited energy shape in the ECAL.
    \begin{description}
    	\item [\textsigma\textsubscript{i\texteta i\texteta} :] Energy weighted spread within the 5x5 crystal matrix centred on the crystal with the largest energy deposit in the supercluster. Obtained by measuring position by countinig crystals. \\
		$ \sigma_{i \eta i \eta} = \sqrt{\frac{\sum^{5x5}_{j}\omega_j (i \eta_j - i \eta_{seed})^2}{\sum^{5x5}_{j}\omega_j}}$ \\
		$i \eta$ is the crystal index at position \texteta and $\omega_i$ is a weight representing the expected energy deposit measured.\\
		$\omega_i = b + ln(\frac{E_i}{E_{5x5}})$
		\item [\textsigma\textsubscript{i\textphi i\textphi} :] same variable as $ \sigma_{i \eta i \eta}$ but computed in the \textphi direction.
		\item [\textsigma\textsubscript{i\texteta i\textphi} :] is the covariance between $ \sigma_{i \eta i \eta}$ and $ \sigma_{i \phi i \phi}$
    \\
		\item [\texteta\textsubscript{width} \: \textgamma :] Shower width in \texteta
    \item [\textphi\textsubscript{width} \: \textgamma :] Shower width in \textphi
	\item [R\textsubscript{9} \: \textgamma :] Energy sum of the 3x3 crystals centred on the most energetic crystal in
    the supercluster divided by the supercluster's energy. Lower values of R\textsubscript{9} for converted photons than those of unconverted photons.
	\item [Had/Em :] Hadronic calorimeter energy deposit over Electromagnetic calorimeter energy deposit
    \item [E\textsubscript{nxm}/E\textsubscript{5x5} :] Energy of most energetic nxm crystal set over energy of 5x5 crystal set
    \end{description}
    \item [\textrho :] Pile-up energy, median of the transverse energy density per unit area.
\end{description}

%%% Local Variables: 
%%% mode: latex
%%% TeX-master: "isae-report-template"
%%% End: 
